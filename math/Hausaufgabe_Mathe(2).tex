\documentclass{article}





\usepackage{microtype}
\usepackage[ngerman]{babel}
\usepackage{amsmath}
\usepackage[dvipsnames]{xcolor}
								%ich weiss auch nicht ganz in welcher reinfolge
								%die pakete müssen

\begin{document}
	In general, we have the following formula.

	\textcolor{red} {\textbf{The Binomial Theorem}}

	\vspace{5mm} 							%5mm vertical space
	If \textit{k} is a positive intiger, then
	
								%gerne feedback wie man space besser macht
	
	\[
		(a+b)^k = a^k + ka^{k-1} b+\frac{k(k-1)}{1\cdot2}a^{k-2} b^2
	\]
	\[
	\hspace*{2.3cm}   	+\frac{k(k-1)(k-2)}{1\cdot2\cdot3} a^{k-3} b^3 + ...
	\]
	\[
	\hspace*{2.5cm}	+\frac{k(k-1)...(k-n+1)}{1\cdot 2\cdot 3 ...\cdot n } a^{k-n} b^n 
	\]
	\[
	\hspace*{0.6cm}	+ ... + kab^{k-1}+b^k
	\]								%geht das auch schöner?? mit hpsace?
									%wie mache ich den zeilenabstand geringer??
									%warum kann man in math kein zeilen %leerlassen? 
									
									
	\textcolor{BlueGreen}{\textbf{Or Summarized}}
	\[
	(a+b)^k = \sum_{n=0}^{n} \binom{n}{k} a^{n-k}\cdot b^k
	\]
	
	
	\textcolor{blue}{\textbf{Example 13}}  Expand $(x-2^5)$
	
	\textbf{Solution} \quad Using the Binomial Theorem with $a = x, b = -2, k = 5$, we have
	
	\[
		(x-2)^5 = x^5 + 5x^4(-2)+\frac{5\cdot4}{1\cdot2}x^3(-2)^2+\frac{5\cdot4\cdot3}{1\cdot2\cdot3}
		x^2(-2)^3+5x(-2)^4+(-2)^5
	\]
	\[
	\hspace*{-3.2cm}	=x^5-10x^4+40x^3-80x^2+80x-32
	\]
			%auch großer Zeilenabstand: Wie mache ich das besser?
\end{document}