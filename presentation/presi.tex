\documentclass[professionalfonts]{beamer}
%\usetheme{...}  % Falls Sie ein spezielles Layout möchten
\usetheme{CambridgeUS}


\usepackage{libertinus}
\mode<presentation>{}
\definecolor{UniBlue}{RGB}{83,121,170}

\title{Pr\"asentation: Latex}
\subtitle{Freie Universit\"at Berlin}
\author{Konstatnin Bobenko}


\begin{document}
	\begin{frame}
		\titlepage
	\end{frame}

{
	\setbeamercolor{background canvas}{bg=UniBlue}
	\begin{frame}
		\begin{itemize}
		\item{Aufgabe a)}
		\item{Kurs: ABV \LaTeX }
		\item{von Herbert Voss}
		\end{itemize}
	\end{frame}
}

	\begin{frame}
		\begin{itemize}
		\item<1-> Dieser 
		\item<2-> Dieser Satz 
		\item<3-> Dieser Satz ist  
		\item<4-> Dieser Satz ist ohne  
		\item<5-> Dieser Satz ist ohne besonderen  
		\item<6-> Dieser Satz ist ohne besonderen Sinn, 
		\item<7-> Dieser Satz ist ohne besonderen Sinn, also  
		\item<8->  Dieser Satz ist ohne besonderen Sinn, also redundant 
		\end{itemize}
	\end{frame}

{
	\setbeamercolor{background canvas}{bg=UniBlue}
	\begin{frame}
		\begin{itemize}
			\item{Aufgabe b)}
			\item{Kurs: ABV \LaTeX }
			\item{von Herbert Voss}
		\end{itemize}
	\end{frame}
}

	\begin{frame}
		\begin{itemize}
			\item<1-> redundant 
			\item<2-> also redundant 
			\item<3-> Sinn, also redundant 
			\item<4-> besonderen Sinn, also redundant 
			\item<5-> ohne besonderen Sinn, also redundant
			\item<6-> ist ohne besonderen Sinn, also redundan
			\item<7-> Satz ist ohne besonderen Sinn, also redundant
			\item<8-> Dieser Satz ist ohne besonderen Sinn, also redundant 
		\end{itemize}
	\end{frame}
{
	\setbeamercolor{background canvas}{bg=UniBlue}
	\begin{frame}
		\begin{itemize}
			\item{Aufgabe c)}
			\item{Kurs: ABV \LaTeX }
			\item{von Herbert Voss}
		\end{itemize}
	\end{frame}
}
	
	\begin{frame}
		\begin{itemize}
			\item \pause[8]Dieser\pause[7] Satz \pause[6] ist \pause[5] ohne \pause[4] besonderen\pause[3] Sinn, \pause[2]also \pause[1] redundant
		\end{itemize}
	\end{frame}

\end{document}